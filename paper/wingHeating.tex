\subsubsection{Wings and Leading Edges [Alberto Marin Cebrian]}
All the temperatures have been calculated for the steady state and for the most demanding operating conditions (cruise, Mach 6.5). The materials used in these exterior surfaces must tolerate high temperatures and a high emissivity is desirable. A higher emissivity will allow the material to reduce the maximum temperature it will have to withstand because it will be able to radiate more heat to the surroundings.

\paragraph{Inlet Cone Tip and Wing Leading Edges}
Infinitely thin surfaces and points are not possible to manufacture. Due to this fact a small normal shock will appear in front of these blunt bodies. This detached normal shock will create a huge deceleration of the flow and the properties of the flow will change significantly, it has been assumed that the air properties ($\gamma$ in particular) will also change as the flow crosses the shock.

Properties after the shock allow us to calculate the Prandtl number at that location

\begin{equation}
Pr=\frac{(\mu_e Cp_e)}{K_e} =0.4574
\end{equation}

Assuming that the boundary layer remains laminar, a similarity solution exists.
\begin{equation}
\frac{du_e}{dx}=\frac{1}{r_n}\sqrt{\frac{2(p_e-p_0)}{\rho_e}}
\end{equation}

The recovery temperature depends on the Prandtl number. For laminar flows the recovery factor is r=sqrt(Pr)
\begin{equation}
T_r=T_e (1+(\gamma_e-1)/2 \sqrt{Pr} M_e^2 )=2082 K
\end{equation}

Convective heat flux on the inlet cone tip
The shape of the blunt cone is approximated to a semi-sphere. The heat flux is given by:

\begin{equation}
\dot{q}=0.763Pr^{0.6}(\rho_e\mu_e)^{1/2}\sqrt{\frac{du_e}{dx}}Cp_e(T_r-T_w)
\label{eqn:tipHeatFlux}
\end{equation}

Convective heat flux on the wing leading edge
The leading edge of the wing has a cylindrical shape with a nose radius equal to half of the thickness of the wing. The heat flux is also given by given by Equation \ref{eqn:tipHeatFlux}.

Radiation heat flux
The radiation heat flux has been calculated with the Stefan-Boltzman equation.

\begin{equation}
\dot{q}_{rad}=\sigma_s (\eta_w T_w^4-T_0^4)
\end{equation}

The steady state solution is reached when $\dot{q}=\dot{q}_{rad}$.

\begin{center}
\begin{tabular}{l c c}
& Inlet Cone Tip & Wing Leading Edge \\
\hline
Nose Radius (m) & $1*10^{-3}$ & $1.27 * 10^{-2}$ \\
Temperature (K) & 1962 & 1435 \\
Material & Tantalum & Nichrome \\
Maximum Operating Temperature & 3023 & 1573 \\
Emmissivity & 0.225 & 0.89
\end{tabular}
\end{center}

The cone tip radius value has been estimated and based on the manufacturing tolerance it may change significantly. On the one hand, a bigger nose radius will decrease the temperature of the cone tip but it will decrease the performance of the inlet adding more drag. On the other hand, a smaller nose radius will be beneficial for the inlet performance but it will increase the temperature of the inlet cone. This value must be small enough not to affect significantly the inlet performance but big enough to be manufactured and get a temperature that lies inside the operating range of the material.

\paragraph{Wing Surface Temperature}
In order to obtain the temperature field over the wing of the aircraft the following assumptions have been made.
	The angle of attack of the vehicle is very small during the flight. This fact allows us to simplify the physics of the problem assuming that the angle of attack ($\alpha$) is zero. In reality the angle will be different from zero and a compression oblique shock wave will appear for the lower part of the wing and an expansion wave will be developed in the upper surface. These waves will be very weak and will not change significantly the results of this analysis as soon as the angle of attack keeps being small enough.
	The wings are long and slender. Flat plate approximation has been used to calculate the heat fluxes at different sections of the wing.
	1-D flow. The velocity of the flow can be simplified to $\overrightarrow{v}=(u,0,0)$.
	
Some important non-dimensional parameters that control the physics of this problem are the Reynolds number, the Prandtl number and the Nusselt number.
The Reynolds number relates the inertial and viscous effects. It determines the change from laminar to turbulent flow in the boundary layer. For a flat plate, the transition Reynolds number from laminar to turbulent is 500,000.
\begin{equation}
\begin{split}
Re_x=\frac{\rho u x}{\mu}	
If   Re_x<500,000, \text{  Laminar boundary layer} \\
	If   Re_x>500,000, \text{  Turbulent boundary layer}
\end{split}
\end{equation}

The Prandtl is a dimensionless parameter representing the ratio of diffusion of momentum to diffusion of heat in a fluid. [2]

\begin{equation}
Pr=\frac{\mu Cp}{K}
\end{equation}

The recovery factor (r) depends on both, Prandtl number and Reynolds number.

\begin{equation}
\begin{split}
r=\sqrt{Pr} \text{  For laminar B.L.}	\\
r=Pr^{1/3} \text{  For turbulent B.L.}
\end{split}
\end{equation}

The Nusselt number is the ratio of convective hat transfer with respect to conduction. The Nusselt number depends on the other two dimensionless parameters (Prandtl number and Reynolds number).

\begin{equation}
Nu_x=\frac{h_g x}{K}
\end{equation}

\begin{equation}
\begin{split}
Nu_x=0.332\sqrt{Re_x} Pr^{1/3}	\text{  For laminar flows} \\
Nu_x=0.0296Re_x^{0.8} Pr^{1/3}	\text{  For turbulent flows}
\end{split}
\end{equation}

With the Nusselt number and the axial position it is possible to obtain the convective heat transfer coefficient $h_g$.

\begin{equation}
h_g=\frac{K Nu_x}{x}
\end{equation}

Note that this analysis is not valid for x=0 where the leading edge is located ($h_g$ is infinite there).
The convective heat flux is 

\begin{equation}
\dot{q}=h_g (T_r-T_w)
\end{equation}


Steady state is reached when the heat dissipated by radiation equals the heat inflow due to convection $\dot{q}=\dot{q}_{rad}$
In order to get the temperature contour of the wing, the wall temperature of the wing has been calculated in numerous locations along the wing.