\section{Conclusions}

Overall, the use of an RDE on a hypersonic vehicle seems very promising, as it is predicted to deliver twice the Isp of a conventional scramjet. Additionally, the flameholding capabilities inherent to an RDE allow for combustion at higher Mach numbers, extending the range of capability for an RDE. However, there still remains many unanswered questions. For example, there is no good method for controlling the number of detonation fronts present in an RDE, which will effect performance. Additionally, deep throttling of the RDE is necessary to transition between acceleration and cruise conditions. This could cause many combustion stability issues due to the change of detonation cell size as the equivalence ratio changes, and the transients involved with throttling and RDE have not been deeply explored as of yet. 

This design could also use more refined and in depth analysis to verify performance. The isolator was only analyzed using well known correlations, but due to the optimistic Mach ratio of the inlet and isolator system, a full shock train analysis should be performed to better characterize such a design. In the regenerative cooling jackets used on the RDE, the changing properties of supercritical ethylene need to be accounted for in order to verify that the combustor lining will not melt. 

Once the above issues have been addressed, there will no doubt be interest in a larger scale hypersonic vehicle that could, itself, carry a payload. As the design scales up, many new opportunities present themselves. For instance, the fuel mass fraction is expected to increase, which will start to push the weight advantage towards storing fuel at low pressure and using a turbopump system to take the ethylene to supercritical conditions. Additionally, the RDE may not have to throttle at such large extremes, as a similarly sized RDE will still have enough thrust to power larger vehicles.