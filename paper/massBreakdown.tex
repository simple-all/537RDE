\section{Mass Breakdown}

To estimate the mass of the vehicle, multiple approaches were used. For generic structures, such as aerodynamic surfaces, the fuselage, and internal support, the weights were scaled by aerodynamic reference area from the results presented in the AFRL GHV concept. The avionics mass does not scale with vehicle size, and so was taken directly from the AFRL paper \cite{ghv}. Specific component masses, such as the isolator and combustor, were estimated based on the geometry and materials used. The resulting mass breakdown can be seen in Table \ref{tab:mass}.

\begin{center}
\begin{tabular}{l c c}
System & Mass (kg) & Mass Fraction (\%)\\
\hline
Fuselage \& Support & 70.2 & 24.2\\
Tanks (Empty) & 3.2 & 1.1\\
Ballast & 18.0 & 6.2 \\
Fuel & 21.0 & 7.2 \\
Avionics & 133.6 & 46.1 \\
Inlet & 7.1 & 2.5 \\
Isolator & 14.3 & 4.9 \\
Compression Duct & 7.4 & 2.5 \\
Combustor & 13.8 & 4.7 \\
Nozzle & 1.6 & 0.5 \\
\hline
\textbf{Total} & \textbf{290.2}
\end{tabular}
\captionof{table}{Mass Breakdown}
\label{tab:mass}
\end{center}