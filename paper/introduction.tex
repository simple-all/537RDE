\section{Introduction}
Sustained air-breathing hypersonic flight has been the focus of a great deal of research over the past several decades. In addition to the high temperatures associated with flight speeds higher than Mach 5, it is difficult to slow down flow speed to subsonic speeds (such as in a ramjet) for sustainable combustion without significant pressure losses, making it necessary to design combustors that can burn at supersonic velocities. One proposal for improving supersonic combustion is to modify the mechanism of reaction all together. Deflagrations require sufficient mixing of fuel and oxidizer to initiate a reaction as well as transfer energy from that reaction to the incoming flow. This is simple to accomplish at low flow speeds but deflagration proves to be an unreliable reaction mechanism at supersonic velocities. Detonations, on the other hand, are supersonic waves that flow through combustible material, reacting it as it passes \cite{wang}. Rotating Detonation Engines (RDE) have shown great promise over recent years due to their ability to maintain combustion at higher Mach numbers than traditional scramjet engines. By maintaining one or multiple detonation waves within the burner, the combustor is able to produce much higher average chamber pressures \cite{bykovskii}.  Additionally, since detonation reactions occur almost instantaneously, the combustion chamber itself can be shortened compared to traditional supersonic burners. The objective of the current work is to propose a design for a gas-injection RDE scramjet intended to be launched from a GO1 launcher to complete a cruising mission at Mach 6.5. The GO1 is capable of boosting a 600 lb vehicle to Mach 5.5 at 1500 psf of dynamic pressure. The fairing requires the vehicle not exceed 24” in diameter and 120” in length. 